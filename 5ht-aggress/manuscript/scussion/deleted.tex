%We know that a very small proportion of indivuals commit a disproportionately large amount of violent acts, 
%which in turn comprise a disproportionately large burden on our criminal justice systems.
%If is possible that some of these individuals violence could be mitigated through targeted pharmotherapeutic interventions,
%it might proove to be a very cost-effective and exceedingly beneficial to society at large.
%However, understanding the general role of 5-HT in aggression is still critically important,
%as it holds enormous potential in addressing a number of problems related to violence.
%To start, we know that a very small proportion of indivuals commit a disproportionately large amount of violent acts, 
%which in turn comprise a disproportionately large burden on our criminal justice systems.
%If is possible that some of these individuals violence could be mitigated through targeted pharmotherapeutic interventions,
%it might proove to be a very cost-effective and exceedingly beneficial to society at large.
%One criminal justice intervention which stems directly from the hypothesized link between 5-HT and aggression involves
%the administration of dietary supplements such as the Omega-3 fatty acids DHA and EPA, which have been shown to affect 5-HT synthesis.
%In fact, a British study that examined the impact of giving prisoners dietary supplements revealed a startingly reduction
%in the number of violent incidents for prisoners who received the supplements as opposed to a placebo.
%This finding was recently replicated in the Netherlands with similar results.
%Compared to the medical costs incurred from violence that occurs \emph{within} prisons,
%the cost of administering dietary supplements is marginal. 
%Despite such encouring findings, it is not clear whether the effects of these supplements were directly due to 
%alterations in 5-HT functioning.
% Getting Right into the evidence.


% QUICK OVERVIEW OF STUDY
% Despite numerous advances in our understanding of serotonin and aggression, the nature of their relationship to each other is still not clear. The serotonin deficiency hypothesis of human aggression has remained the most prominent hypothesis of their relation for the last several decades; however, this hypothesis has failed to account for a number of contradictory findings. Substantively, the hypothesis has remained largely static and has not evolved alongside our understanding of serotonin's complexity. The continued prominence of this hypothesis may be partially responsible for previous failures to clarify the serotonin--aggression link in humans.


% It is probably that the serotonin-aggression link cannot be adequately examined without putting it the context of serotonin's effects on other clinical phenomena. 
%
%
% One answer comes from personality researchers who have identified a set of overlapping and \emph{jangled} personality constructs (\emph{neuroticism, negative affectivity, emotional instability, harm avoidance, vulnerability, stress reactivity, etc.}) that refer in varying degrees to a trait-level tendency to experience marked levels of psychological and emotional distress \parencite{Ormel2004}. Thus, it is possible that the serotonin--aggression link is mediated by a serotonergic-deficit-induced neuroticism or increased emotional reactivity and tendency to experience negative affect. Further, the construct of neuroticism may help explain the disparity we found between self-perceived and other-perceived aggression.  Neuroticism is linked less to behavioral aggression than it is to angry and aggressive feelings \parencite{Caprara1996}. Recent research suggests that neurotic\slash emotional aggression more often expresses itself internally as opposed to externally \parencite{Egan2011, Settles2012} and is often inversely related to other-directed expressions of aggression \parencite{Donnellan2005}. When neuroticism is linked to other-directed aggression, it is typically associated with passive or indirect forms of aggression as opposed to more overt, direct forms of aggression, which are more often associated with other personality constructs such as extraversion and disagreeableness \parencite{Edmunds1977, Egan2009}. This is intuitive if one considers that neuroticism envelops constructs such as self-esteem and self-consciousness: individuals with low self-esteem are more likely to avoid confrontation while individuals with inflated levels of self-esteem (low neuroticism) may be more uncompromising in their interactions with others \parencite{Baumeister1996}.
% 
% Another explanation of the effects of serotonergic dysfunction on aggression, anger, and hostility may be found in yet a broader 
% examination of serotonin's role in the human brain.






% The last major limitation that we will address involves the difficulty we faced in testing the relation between serotonin and the various subtypes of aggression, anger, and hostility. Originally, we hoped to code meaningful subtypes of these constructs based upon underlying motivation, affective content, mode of expression, and directness. However, in the process of coding studies on serotonin and aggression, it quickly became apparent that this would prove problematic because many measures of aggression fail to distinguish between  different aggression-related constructs and their subtypes.  For example, the most commonly used instrument used to measure aggression included in the current meta-analysis is the \emph{Buss-Durkee Hostility Inventory} \parencite[BDHI;][]{Buss1957}, which includes subscales labeled \emph{Assault, Indirect Hostility, Irritability, Negativism, Resentment, Suspicion, and Verbal Hostility}. In some cases, subscales more related to expression of aggressive behavior were reported (i.e., Assault and Verbal Hostility), while in other instances, only the total score was given. Further, subscale interpretation for this measure is compromised by questionable reliability \parencite{Vassar2009} and discriminant validity \parencite{Biaggio1981}. Initial \parencite{Buss1957} and subsequent \parencite[see][]{Bushman1991} factor analytic studies have identified a two-factor structure for the BDHI with an ``emotional'' (aka ``neurotic'') factor and a ``motor'' (aka ``expressive'') factor. Hence, the BDHI total score uses ``hostility'' as the umbrella term to encompass the affective (i.e., ``emotional''), cognitive (i.e., ``neurotic''), and behavioral (i.e., ``motor'' or ``expressive'') aspects of aggression. There is also some evidence that interpretation of the BDHI total score suffers from a high level of distortion due to social desirability \parencite{Biaggio1980}. The example of the BDHI demonstrates the difficulty in separating out conceptually distinct (although highly intercorrelated) aspects of aggression.


% , with the two receptor families most often linked to human aggression are the 5-HT\subscript{1} and 5-HT\subscript{2} families;
% however, several preclinical studies have implicated 5-HT\subscript{3} in the regulation of aggressive behavior as well \parencite[e.g.,][]{Carrillo2010, Cervantes2009, Cervantes2010, Ricci2004, Ricci2005}. Interpretation of a central serotonin deficit's impact on aggressive behavior is complicated by the fact that 


% The following four recommendations for future research roughly correspond to the 
% In systematically evaluating the evidence for the serotonin deficiency hypothesis of human aggression, we have come up with a number of recommendations that we believe will further advance this important field of research.


% find: \\cite\<(.*)\>{               replace: \parencite[\1][]{
% \\cite\{                 \parencite{
% \\cite\[(.*)\]\{         \parencite[][\1]{
% \\cite\<(.*)\>\[(.*)\]\{         \parencite[\1][\2]{
% \\citeA\{                \\textcite{













% will be limited (i.e., ``small'').
% There are two reasons that lead us to believe that a small effect is more likely than a null effect despite the concerns just mentioned. First, there is reason to believe that serotonin does play a \emph{complex} role in the regulation of aggressive behaviors \parencite[e.g.,][]{Quadros2009}, therefore many studies are likely to find some sort of an effect (whether negative or positive). Second, there is considerable evidence that meta-analyses tend to systematically overestimate population effect sizes 
% There is considerable evidence that meta-analyses tend to systematically overestimate population effect sizes \parencite[e.g.,][]{Burdett2003, Levine2009, Sutton2000}; thus, given the early and continued prominence of the serotonin deficiency hypothesis in this body of literature, studies finding positive or nonsignificant correlations would be expected to have a more difficult time getting published. 



% \subsection{Aggression}
% \begin{quotation}\noindent 
% ``\emph{Aggression ranks among the most misunderstood concepts in all the behavioral sciences}'' \parencite[][p. 1]{Huber2011}.
% \end{quotation}
% 
% 
% % DEFINING AGGRESSION
% Any discussion of human aggression should commence with a definition.
% A commonly used definition is provided by Baron and Richardson who define aggression as 
% ``any form of behavior directed toward the goal of harming or injuring another living being who
% is motivated to avoid such treatment'' \parencite[][p. 7]{Baron1994}.
% This definition was expanded upon by Geen to include ``intent to harm'' and an 
% ``expectation of causing such harm'' \parencite[][p. 3]{Geen2001}.
% We adopt a similar definition for the purposes of this review ---
% that human aggression represents any volitional efforts to harm another person.
% 
% 
% Such a broad definition encompasses an enormous range of acts from the relatively benign 
% (e.g., a schoolyard taunt) to the disturbingly malignant (e.g., aggravated murder). 
% Therefore, it is useful to have a taxonomy of human aggression that reflects the most
% important qualitative distinctions among aggressive behaviors.
% Unfortunately, this is a nontrivial task given such a heterogeneous construct.
% Currently, there is no preferred taxonomy of human aggression subtypes. 
% Numerous categories have been proposed based upon factors such as motives 
% (Anderson \& Bushman, \citeyear{Anderson2002c}),
% modes of expression \parencite{Parrott2007},
% degree of forethought \parencite{Barratt1998},
% and emotionality \parencite{Vitiello1990}.%
% \footnote{For a more detailed treatment of human aggression subtypes, 
% see \nptextcite{Parrott2007} and \nptextcite{Ramirez2006}.}
% However, in practice, it is often difficult to make clear distinctions between 
% the many proposed subtypes of aggression.
% Indeed, it is often difficult to distinguish aggression itself from the related 
% constructs of hostility and anger.
% 
% 



% Finally, we discuss four ways to interpret the findings and provide four recommendations for advancing this important area of research.
% ) while critically examining three core assumptions underlying the study of the serotonin deficiency hypothesis of human aggression:
% 1) that human serotoninergic functioning can be measured reliably and validly, 
% 2) that changes in aggregate serotonin levels have a similar patterns of effects across inviduals, and
% 3) that human aggression can be measured in a reliable and valid way.
% Finally, we present a comprehensive meta-analysis designed to test whether divergent findings in the literature 
% can be explained by taking into account methodological factors 
% (serotonin assay method, aggression measure, study design, sample size, etc.) and demographic variables
% (gender, age, history of aggression, clinical status, etc.).



% Cerebrospinal fluid (CSF) Concentrations of serotonin's principle metabolite, 5-HIAA, are thought to index cerebral serotonergic activity \parencite{Lesch03, Tuinier96}. CSF 5-HIAA has long been hypothesized to predict aggression \parencite[e.g.,][]{Asberg76, Asberg762, Bioulac78, Brown79}. However, as is clear in the quotation above, there have been concerns about over-interpreting this relation.



% \subsection{Recommendations}
% We have four recommendations for advancing our understanding of the relationship between serotonin and human aggression. 
% % \subsubsection{1. Assess and Improve Measurement Reliability and Validity}
% Our first recommendation follows that made by Tuinier et al. in their review of serotonin and disruptive behaviour published over a decade and a half ago, which stated that ``there is still a need to define the manifestations of aggression more clearly, to develop more reliable assessment methods and to study whether, as in animals, subforms of aggression are distinguishable'' (1996, p. 479). Further, there is a need to systematically assess and improve the reliability of our 
% 
% Human aggression, anger, and hostility are multi-faceted constructs that can manifest in vastly different ways. The studies on serotonin and aggression reviewed herein suggest that this concern has not received enough attention among researchers interested in the interplay between serotonin and human behavior. This may be partly due to the ambiguous prescriptions of the simple serotonin deficiency hypothesis of aggression with regards to how serotonin deficiency might differentally effect subtypes of aggression.


% \subsubsection{2. Recognize Contradictory Findings}
% Our second recommendation is to ...


% \subsubsection{3. Test for Interactions}
% Our third recommendation is to ...
% Our findings may be interpreted by some as weak evidence for the serotonin deficiency hypothesis of aggression; however, it should be clear to most that the actual relationship between serotonin and aggression is not a simple inverse correlation, but a complex and interactive relation involving multiple neurobiological systems, neurotransmitters, and functionally distinct serotonin receptor types.


% \subsubsection{4. Revise the Serotonin Deficiency Hypothesis}
% Our fourth and final recommendation is too...
% Our final recommendation follows the first two: 
% we hope that researchers will stop conducting \emph{simple} serotonin-aggression research. 
% Simple serotonin aggression research involves viewing aggression as a unitary, simple construct, or treating serotonin as an isolated. We have been encouraged by the trend towards researching relations between multiple neurotransmitter/endocrine endpoints and multiple human social behaviors and traits. This transition can be accelerated by continuing to highlight the limitations of the simple serotonin deficiency hypothesis of human aggression. Over-reliance on the simple serotonin deficiency hypothesis of aggression has come at a cost -- theory has not kept pace with our advancements in empirical knowledge.

% SUMMARIZE 
% These complexities are not consistent with the presumed mechanistic simplicity of the serotonin deficiency hypothesis of aggression, and for this reason, many researchers have questioned its utility or challenged its over-simplifications \parencite[e.g.,][]{Boer2005, Booij2010, Carrillo2009, Tuinier96}.




% Genetics studies have also had mixed results. 
% \footnote{Several candidate genes related to serotonin metabolism, morphology, and transport have been studied in the context of aggressive behavior \parencite{Gunter2010} 
% %
% % FOOTNOTE
% \footnote{Several candidate genes related to serotonin metabolism, morphology, and transport have been studied in the context of aggressive behavior \parencite{Gunter2010} including the serotonin metabolic genes monoamine oxidase A \parencite[MAOA; e.g.,][]{Kinnally2009, Reif2007, Sjoeberg2008, Weder2009}, tryptophan hydroxylase 1 \parencite[TPH 1;][]{Hennig2005}, and tryptophan hydroxylase 2 \parencite[TPH 2;][]{Mann2008a}, the serotonin receptor genes 5HTR1B \parencite{Soyka2004}, 5HTR2A \parencite{Burt2008, Mik2007}, and 5HTR3B \parencite{Ducci2009}, as well as the promoter region of the serotonin transport gene 5-HTTLPR \parencite[e.g.,][]{Douglas2011, Gerra2004, Liao2004, Reif2007, Retz2004}.} 
% % END FOOTNOTE


% (5-HT\subscript{1A},  5-HT\subscript{1B}, 5-HT\subscript{1D}, 5-HT\subscript{1E}, 5-HT\subscript{1F}, 5-HT\subscript{2A}, 5-HT\subscript{2B}, 5-HT\subscript{2C}, 5-HT\subscript{3}, 5-HT\subscript{4}, 5-HT\subscript{5A}, 5-HT\subscript{6}, 5-HT\subscript{7A}, 5-HT\subscript{7B}, 5-HT\subscript{7D}).


% Furthermore, researchers have long pointed out that nonindependence problems in sampling such as publication bias \parencite[e.g.,][]{Burdett2003, Levine2009, Rosenthal79, Sutton2000, Sutton09} coupled with the lower power and high measurement error of many small-n studies leads to predictably inflated correlations \parencite{Fiedler2011, Vul2009}. 
% Finally, high levels of measurement error coupled with a high rate of small-$n$ studies leaves the serotonin--aggression link particularly vulnerable to publication bias induced inflation \parencite[][]{Burdett2003, Levine2009, Sutton2000}.


% Of note also are the findings by some researchers that human males with low or disrupted monoamine oxidase A (MAOA) expression exhibit increased aggressive tendencies, particularly when raised in hostile environments \parencite{Caspi2002, Foley2004, Frazzetto2007, Kim-Cohen2006, Manuck2000a, Weder2009, Widom2006}. Given MAOA's role in facilitating the metabolism of monoamines such as serotonin, impaired MAOA expression indicates tonically \emph{increased} serotonin levels, which, according to the deficiency hypothesis, should lead to \emph{reduced} aggression.  