% \documentclass[jou]{apa6}
% 
\usepackage[american]{babel}
\usepackage[backend=biber,style=apa,sortcites=true,sorting=nyt]{biblatex}
\DeclareLanguageMapping{american}{american-apa}
\usepackage{csquotes}
\addbibresource{/home/duke/Dropbox/rev5ht/paper/rev5ht.bib}
\usepackage{graphicx}
\usepackage{tikz}
\usepackage{amsmath}
\usepackage{amsfonts}
\usepackage{mathtools}
\usepackage{array}
\usepackage{mathptmx}
\usepackage{framed}
\usepackage{dcolumn}
\usepackage{multirow}
\newcolumntype{d}[1]{D{.}{.}{#1}}
\newcommand{\subscript}[1]{\ensuremath{_{\textrm{#1}}}}
\PassOptionsToPackage{hyphens}{url}\usepackage{hyperref} 
\usepackage{setspace}% http://ctan.org/pkg/setspace
\AtBeginEnvironment{tabular}{\singlespacing}% Single spacing in tabular environment
% http://tex.stackexchange.com/questions/44013/apa6-class-single-spaced-tables-for-manuscript
\usepackage{footmisc}
\renewcommand{\footnotelayout}{\doublespacing}
\definecolor{mygreen}{RGB}{0,87,13}
% \hypersetup{
% colorlinks=true,
% citecolor=mygreen,
% bookmarksopen=true
% }
\hypersetup{
colorlinks=true,
linkcolor=black,
citecolor=black,
filecolor=black,
urlcolor=black,
bookmarksopen=true
}


% ,
% pdfinfo={
%  Title={Bayesian Analysis of Parents' Drinking Motives and Their Children's Adjustment},
%  Author={Aaron A. Duke, M.S.}
% }, 
% pdfdisplaydoctitle=true
% }

% 1.   One minor suggestion to the authors might be “a collective within-lab factor analysis of constructs” prior to z transformation. However I do not believe this effort is necessary for the publication of this analysis.

\author{~} 
% \affiliation{\vspace{1mm}Department of Psychology\\University of Kentucky}
\affiliation{~}
\journal{Psychological Bulletin}
\volume{Volume 0, Number 1}
\ccoppy{\copyright ~2012 Aaron Duke}
\copnum{aaron.duke@uky.edu}
\ifapamodejou{ % JOU MODE
\note{Draft Date: \today{}} 
}{}
% \authornote{\hspace{-1.3em}
% \emph{Corresponding Author:}\\
% Aaron A. Duke \\
% Department of Psychology,
% University of Kentucky\\
% Lexington, KY 40506-0044, USA\\
% Email: \href{mailto:aaron.duke@uky.edu}{aaron.duke@uky.edu}
% }

% \title{Revisiting the Serotonin Deficiency Hypothesis\\ of Human Aggression: A Meta-analysis}
\title{Revisiting the Serotonin-Aggression Relation in Humans: A Meta-analysis}
% \title{The Decline Effect in Action: A Meta-analytic Review of Serotonin and Human Aggression}
\shorttitle{Serotonin and Human Aggression Revisited}


\abstract{%
The inverse relation between serotonin and human aggression is often portrayed as `reliable,' `strong,' and `well-established' despite decades of conflicting reports and widely recognized methodological limitations. In this systematic review and meta-analysis, we evaluate the evidence for and against the serotonin deficiency hypothesis of human aggression across four methods of assessing serotonin: cerebrospinal fluid levels of 5-hydroxyindoleacetic acid (CSF 5-HIAA), acute tryptophan depletion, pharmacological challenge, and endocrine challenge. Results across 175 independent samples and over 6,500 total participants were heterogeneous, but, in aggregate, revealed a small correlation between serotonin and aggression, anger, and hostility, $r$ = -.12. Potential methodological and demographic moderators largely failed to account for variability in outcomes. We discuss four possible explanations for the pattern of findings: 1) unreliable measures, 2) ambient correlational noise, 3) an unidentified higher-order interaction, and 4) a  selective serotonergic effect. Finally, we give four recommendations for advancing this important area of research: 1) acknowledge contradictory findings and avoid selective reporting practices,  2) focus on improving the reliability and validity of serotonin and aggression measures, 3) test for interactions involving personality and\slash or environmental moderators, and 4) revise the serotonin deficiency hypothesis to account for serotonin's functional complexity. 
} 
\keywords{serotonin, 5-HT, aggression, anger, hostility}

% Early evidence suggesting a possible link between serotonin and aggression 
% has sparked decades of exciting research into the neurochemical substrates of human aggression.
% With advances in measurement technology and a growing body of evidence, it has become clear 
% that previous notions of a simple deficiency in central serotonin levels acting as a catalyst 
% for aggressive and violent behavior are inadequate. 
% In spite of this, many studies continue to be designed around the several-decades-old theory 
% that a general deficiency in central serotonin levels leads people to be more aggressive. 
% The current meta-analysis reviews studies across four distinct methods of assessing central 
% serotoninergic functioning in humans (cerebrospinal fluid 5-HIAA, acute tryptophan depletion, 
% pharmacological challenge, and endocrine challenge) in order to empirically assess whether the 
% serotonin deficiency hypothesis of human aggression is valid. 
% Results across 175 independent samples and over 6,500 total participants were heterogeneous, 
% but, in aggregate, revealed a ``small'' correlation between serotonin and aggression, anger, and hostility,
% $r$ = -.12. 
% Numerous methodological and demographic variables largely failed to account for variability in outcomes. 
% We argue that researchers should move beyond the simple deficiency hypothesis in examining the effects of 
% serotonergic neurotransmission on aggression and that the continued prominence of this hypothesis may be partially 
% responsible for previous failures to clarify the serotonin--aggression link in humans. 
% Implications and possible explanations of some unexpected meta-analytic findings are discussed, 
% including the finding that other-report\slash observational measures of aggression 
% were \emph{positively} and significantly correlated to serotonergic functioning, $r$ = .17.
% \begin{document}
% \maketitle


\section{Discussion}
%%%%%%%%%%%%%%%%%%%%%%%
% SUMMARY OF FINDINGS %
%%%%%%%%%%%%%%%%%%%%%%%
The current meta-analysis was designed to critically evaluate the serotonin deficiency hypothesis of human aggression by reviewing findings across four methods commonly used to assess central serotonergic functioning. We hypothesized that overall, the inverse correlation between serotonin and aggression, anger, and hostility would be small because of concerns over measurement reliability and the inherent complexity of cerebral serotonergic functioning. The results of the meta-analysis, which included 175 independent samples and over 6,500 total participants, supported our primary hypothesis. The combined weighted correlation between serotonin and aggression, anger, and hostility was $r = -.12$, which maps very closely onto Cohen's convention for a ``small'' effect size (1992, p. 157). Thus, in contrast to Brown and colleague's original finding that CSF 5-HIAA could account for 80\% of the variance in a group of soldiers' aggression \parencite{Brown79}, we have found that serotonin explains just over 1\% of variance in aggression, anger, and hostility. This contrast is even more notable when the comparison is constrained to CSF 5-HIAA measurement only, for which the weighted mean effect size was not significantly different from zero when using a random effects model. As is clear in Figure~\ref{fig:hiaa_forest}, the magnitude of Brown et al.'s original finding correlating CSF 5-HIAA and human aggression has not been replicated despite more than 30 attempts conducted across a period spanning over three decades.\footnote{See \nptextcite[see][]{Lehrer2010} for other examples of the `decline effect.'} 


% BRIEF OVERVIEW OF OTHER FINDINGS
Our prediction that findings would be highly heterogeneous was only partially supported by the results. Significant heterogeneity was found \emph{within} each method used to assess serotonergic functioning except pharmacological challenge and cortisol challenge. Contrary to expectations, significant heterogeneity \emph{between} methodologies used to test serotonergic functioning was not found. Numerous methodological and sample-level variables were tested to see if they could explain heterogeneous findings. While in a few cases, methodological variables did explain some of the variance within certain assessment methods, there was no consistent pattern of findings. On the other hand, none of the sample-level variables, including percent female, mean age, history of aggression, and the presence of clinical psychopathology, were significantly related to the direction or magnitude of study effect size. Significant heterogeneity remained in all the models even after including potential methodological and sample-level moderators. While these results should not be taken as conclusive evidence that these factors are not important in moderating the influence of serotonin on aggression, anger, and hostility, this study-level analysis suggests that they do not provide an adequate explanation of divergent findings linking serotonin to aggression.


\subsection{Alternative Interpretations}
The relation between serotonin and human aggression as well as the validity of the serotonin deficiency hypothesis remain open to debate. The small, inverse correlation between serotonergic functioning and human aggression found in the present meta-analysis can be interpreted as either supporting or refuting the serotonin deficiency hypothesis. We briefly address four possible interpretations for our findings below. 
% (the effect was significantly different from zero)  (the effect was quite small overall)


\subsubsection{1. The relation was attenuated by a lack of reliable and valid measures}
We have already addressed some of the reliability and validity concerns relevant to measuring human serotonergic functioning; however, the reliability and validity of measures on the other side of the equation (i.e., measures of aggression, hostility, and anger) are also important to consider. Unreliability in serotonin and/or aggression measures could lead to attenuated correlations in spite of the existence of an actual relation \parencite{Buckley1990}. 
% One method to potentially improve reliability is to distinguish between various subtypes of aggression, anger, and hostility across dimensions such as motivation, affective content, mode of expression, and directness. However, few measures of aggression distinguish adequately between different aggression-related constructs, not to mention their subtypes. This difficulty is compounded by inconsistent reporting of subscales \parencite{Vassar2009} and operationalization of measures \parencite{Ferguson2012}. 
For example, the subscales of the most commonly used aggression instrument included in this meta-analysis, the \emph{Buss-Durkee Hostility Inventory} \parencite[BDHI;][]{Buss1957}, have been found to lack adequate reliability \parencite{Vassar2009} and discriminant validity \parencite{Biaggio1981}. Further, there is concern that self-report survey instruments tend to be only weakly correlated with observable aggressive behavior \parencite{Ferguson2009a, Gothelf1997}. We found that the overall relation between serotonergic functioning and aggression was reversed in \emph{other-report} aggression measures (including observational instruments). In other words, observable aggression was positively associated with serotonin levels while self-reported aggression was negatively associated with serotonin levels. 
% Finally, the reliability and validity of laboratory aggression measures has been repeatedly questioned \parencite[e.g.,][]{Ferguson2009b, Ferguson2008a, Ritter2008, Tedeschi1996, Tedeschi1998} and may also have hindered researchers' ability to detect a larger underlying effect if it indeed existed.


\subsubsection{2. The relation is due to ambient correlational noise}
``Ambient noise'' (Lykken, \citeyear{Lykken1968}; referred to as the ``crud factor'' by Paul Meehl;  \citeyear{Meehl1990a}) refers to the observation that in the social sciences, and to some degree, the biological sciences, ``everything correlates to some extent with everything else'' \parencite[][204]{Meehl1990a}. That studies of serotonergic functioning may have considerable ambient correlational noise should not be terribly surprising given the observation that ``serotonergic neurotransmission contributes to the regulation of virtually every neurobiological function, including sensory, motor, cognitive, and higher order executive functions'' \parencite[35]{Kerman2011}. Inflated correlations are prevalent in the social sciences \parencite[see][]{Fiedler2011}, and it is possible that the serotonin deficiency hypothesis has been reinforced in large part by publication bias \parencite{Burdett2003, Levine2009, Rosenthal79, Sutton2000, Sutton09}.


\subsubsection{3. The relation is moderated by a higher-order function of serotonin}
Serotonin dysfunction has been implicated in a large number of psychopathological conditions and traits that have very different behavioral phenotypes \parencite{Deakin2003}. For example, serotonergic deficits have been implicated in pathology as divergent as social anxiety \parencite{Stein2002} and antisocial personality \parencite{Voellm2010}. It is reasonable to assume that serotonergic dysfunction contributes to these disorders at a relatively broad or primitive level given their \emph{prima facie} differences. Therefore, the small correlation between serotonin and aggression observed in the present meta-analysis may be due to a failure to control for some higher-order function of serotonin.


\textcite{Tops2009} argue that the phylogenetically primitive function of serotonin is to modulate the drive to withdraw and seek contentment. They propose that serotonin counteracts the dopaminergic-induced drive to seek stimulation with a drive to withdraw from stimulation. These authors take care to distinguish between \emph{withdrawal} (the drive to reduce stimulation) and \emph{avoidance} (the drive to avoid threat or harm). They suggest that serotonin facilitates the drive to \emph{withdraw} by decreasing the impact of emotional and environmental cues and inducing a ``satiated waking state'' (p. 432). Serotonin dysfunction may reduce an organism's ability to decrease reactivity and increase contentment. A compromised ability to withdraw from aversive stimuli might lead to increased efforts at avoiding (as opposed to withdrawing from) aversive stimuli including unpleasant emotional states such as feelings of rejection or anxiety. The ultimate act of avoidance -- suicide -- has long been implicated in connection to serotonergic dysfunction, albeit with inconsistent evidence \parencite[e.g.,][]{Lester1995}. The earliest human studies linking serotonin and aggression stemmed from observations that depressed inpatients with low CSF 5-HIAA levels attempted more violent forms of suicidal behavior \parencite[e.g.,][]{Asberg76, Asberg762}. If one cannot withdraw in the face of threat, then the remaining options are to confront the threat or avoid it (i.e., fight or flight). Anti-social and socially avoidant behavior represent two fundamentally different response tendencies associated with serotonergic dysfunction. By failing to distinguish between these two response alternatives, we may be masking serotonin's true effects in either one of these domains by aggregating across both. Hence, the limited finding between serotonin and aggression in the current meta-analysis may be due to a missing individual-difference moderator such as the tendency to fight or flight. 
%  However, similar to the aggression research, this literature has been plagued by a number of inconsistent findings \parencite{Lester1995}. 


\subsubsection{4. The relation is specific to certain aspects of serotonergic functioning}
When the serotonin deficiency hypothesis of human aggression was originally conceived, our understanding of serotonergic neurotransmission was much simpler than it is today. Over the past several decades, rapid advances in technology and copious amounts of research have led to a much clearer understanding of serotonin's role in the brain; however, with increased understanding has come an increased awareness of the complexity of serotonin's role in regulating behavior. The implications of a putative deficit in central serotonin functioning have also become less and less clear. We now know that there are ``several anatomically distinct [serotonergic]  pathways'' \parencite[212]{Murphy1996} and at least 7 families of serotonin receptors encompassing well over a dozen different serotonin receptor types. These different receptors are often associated with fundamentally different biological functions or, in some cases, opposite effects on the same function \parencite{Kitson2007}. For example, 5-HT\subscript{1A} and 5-HT\subscript{1B} receptors are often argued to \emph{attenuate} aggressive behavior when activated, while the 5-HT\subscript{2A} and 5-HT\subscript{2C} receptors are thought to \emph{exacerbate} aggression when activated \parencite{Quadros2009}. Truely understanding serotonin's role in the brain requires distinguishing between tonic and phasic effects \parencite[e.g.,][]{Boer2005, Erp2000, Vegt2003}, pre- vs. post-synaptic activity \parencite{Boer2005, Olivier2005}, and the interactive effects on other neurotransmitters \parencite{Olivier2005}. The serotonin deficiency hypothesis does not stipulate why a central deficit should influence some aspects of serotonergic functioning more than others, and it is possible that the countervailing impact of different selective serotonergic effects masks the true extent of the serotonin--aggression relation.


\subsection{Limitations}
The first limitation of this review was that it did not include studies examining serotonin receptor binding potentials or genetic polymorphisms. These methodologies were not included due to concerns over parsimony, feasibility, and interpretability. Studies of binding potentials and genetic polymorphisms are more diverse and fewer in number than studies using the four methodologies included in present meta-analysis. Serotonin receptor binding potential studies attempt to index serotonin receptor \emph{affinity} or \emph{density} by examining peripheral blood platelet binding \parencite[e.g.,][]{Coccaro1996} or regional receptor density using positron emission tomography \parencite{Witte2009}. Similar to other methods of assessing serotonin, binding potential studies have been characterized by inconsistent, contradictory, and null findings \parencite[e.g.,][]{Castrogiovanni1994, Maguire1997, Marazziti1991}. For example, the two published studies utilizing the high 5-HT$_{\textrm{1A}}$ affinity radioligand [carbonyl-$^{11}$C]WAY-100635 to measure the density of 5-HT$_{\textrm{1A}}$ receptors in the prefrontal and anterior cingulate cortices and their relation to aggression found opposite results \parencite{Parsey2002, Witte2009}.\footnote{Interestingly, Witte and colleagues interpreted the opposite findings as supporting different roles for \emph{pre-} and \emph{post-}synaptic 5-HT$_{\textrm{1A}}$ receptors \parencite{Witte2009}.} Studies examining candidate genes related to serotonin metabolism, morphology, and transport \parencite[see][]{Gunter2010} have also had mixed results. For example, the short-allele of 5-HTTLPR has been shown to increase antisocial personality disorder by some researchers \parencite{Douglas2011} and to increase social conformity by others \parencite{Homberg2011}. The full array of inconsistencies in binding potential and genetic studies relevant to the serotonin-aggression link in humans cannot be adequately addressed here, but to date, there has been no definitive evidence from either of these two approaches supporting the serotonin deficiency hypothesis of human aggression. Therefore, it is unlikely that our primary conclusions would have differed substantially with the inclusion of these two additional methodologies.


A second limitation concerns the moderator analyses conducted to assess whether methodological or sample-level variables accounted for significant variance in study outcomes. Poor reporting practices, a common problem in meta-analysis coding \parencite[e.g.,][]{Cox1995c, Lipsey01}, along with measurement issues impeded the resolution with which we could code certain variables (e.g., subtypes of aggression). In several cases, the number of studies having a particular level of a variable was relatively small, leaving open the possibility that some analyses lacked sufficient power. 


\subsection{Conclusion \& Recommendations}
The evidence for the serotonin deficiency hypothesis of human aggression is equivocal. Contradictory findings, unreliable measurement, and a high degree of complexity leave our overall finding of a small inverse correlation between serotonin and human aggression open to multiple, equally plausible interpretations. While the overall relation between serotonin and human aggression is currently unclear, the four recommendations below hold promise in advancing this important area of research and paving the way for future clarity. 


\subsubsection{1. Acknowledge contradictory findings and avoid selective reporting}
Our first recommendation is for researchers to acknowledge the contradictory findings within the literature and within their own research. Selective citation of the evidence has led to a widespread, demonstrably false belief that the link between serotonin and human aggression is ``perhaps the most reliable finding in the history of psychiatry'' \parencite[15]{Fishbein2001}. 
The current system of scientific publishing can incentivize researchers to selectively report findings from their data \parencite[see][]{Bones2012, Ioannidis2005, Molloy2011}. Therefore, we strongly urge researchers to embrace the ``Open Data'' and ``Reproducible Research'' movements and make their data available online for reanalysis, critique, and reuse by other researchers \parencite[see][]{Gelman2011a, Molloy2011, Peng2009, Peng2011, Simonsohn2012}. 


\subsubsection{2. Focus on improving the reliability and validity of aggression and serotonin measures}
Our second recommendation follows that made by Tuinier and colleagues in their review of serotonin and disruptive behaviour published over a decade and a half ago when they stated, ``there is still a need to define the manifestations of aggression more clearly, to develop more reliable assessment methods and to study whether, as in animals, subforms of aggression are distinguishable'' (\citeyear{Tuinier96}, p. 479). We recommend focusing more effort on ensuring that measures of aggression and serotonergic functioning are reliable and valid. It has been observed that advances in theory often follow advances in methodology \parencite{Greenwald2012}; this may very well be the case with respect to serotonin and aggression. 


\subsubsection{3.  Test for interactions involving personality and environmental moderators}
Our third recommendation is to study serotonin and aggression alongside, and in the context of, other psychological constructs such as social avoidance, depression, and anxiety. The idea that serotonin is uniquely related to aggression is presumptuous and may confound different individual response tendencies \parencite[see][]{Deakin2003, Tops2009}, limiting our ability to identify those individuals for which serotonin deficiency is a key component in their aggressive tendencies. 


\subsubsection{4. Revise the serotonin deficiency hypothesis to account for serotonin's functional complexity}
Finally, we encourage researchers to recognize the excessive simplicity of the serotonin deficiency hypothesis. Albert Einstein is attributed to saying that ``everything should be made as simple as possible, but no simpler'' \parencite[475]{Einstein2010}. Given the divergent effects of different serotonin receptor types (e.g., 5-HT$_{\textrm{1A,B}}$ vs. 5-HT$_{\textrm{2A,C}}$; \nptextcite{Quadros2009}) and the ongoing debates over pre- vs. post-synaptic activation \parencite{Boer2005, Olivier2005} and the relative importance of phasic vs. tonic serotonergic activity \parencite{Boer2005, Erp2000, Vegt2003}, it may be time to abandon a nonspecific deficiency model in favor of a specific or selective deficiency model. 


% \printbibliography
% 
% 
% \end{document}


