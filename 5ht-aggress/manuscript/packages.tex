
\usepackage[american]{babel}
\usepackage[backend=biber,style=apa,sortcites=true,sorting=nyt]{biblatex}
\DeclareLanguageMapping{american}{american-apa}
\usepackage{csquotes}
\addbibresource{/home/duke/Dropbox/rev5ht/paper/rev5ht.bib}
\usepackage{graphicx}
\usepackage{tikz}
\usepackage{amsmath}
\usepackage{amsfonts}
\usepackage{mathtools}
\usepackage{array}
\usepackage{mathptmx}
\usepackage{framed}
\usepackage{dcolumn}
\usepackage{multirow}
\newcolumntype{d}[1]{D{.}{.}{#1}}
\newcommand{\subscript}[1]{\ensuremath{_{\textrm{#1}}}}
\PassOptionsToPackage{hyphens}{url}\usepackage{hyperref} 
\usepackage{setspace}% http://ctan.org/pkg/setspace
\AtBeginEnvironment{tabular}{\singlespacing}% Single spacing in tabular environment
% http://tex.stackexchange.com/questions/44013/apa6-class-single-spaced-tables-for-manuscript
\usepackage{footmisc}
\renewcommand{\footnotelayout}{\doublespacing}
\definecolor{mygreen}{RGB}{0,87,13}
% \hypersetup{
% colorlinks=true,
% citecolor=mygreen,
% bookmarksopen=true
% }
\hypersetup{
colorlinks=true,
linkcolor=black,
citecolor=black,
filecolor=black,
urlcolor=black,
bookmarksopen=true
}


% ,
% pdfinfo={
%  Title={Bayesian Analysis of Parents' Drinking Motives and Their Children's Adjustment},
%  Author={Aaron A. Duke, M.S.}
% }, 
% pdfdisplaydoctitle=true
% }

% 1.   One minor suggestion to the authors might be “a collective within-lab factor analysis of constructs” prior to z transformation. However I do not believe this effort is necessary for the publication of this analysis.

\author{~} 
% \affiliation{\vspace{1mm}Department of Psychology\\University of Kentucky}
\affiliation{~}
\journal{Psychological Bulletin}
\volume{Volume 0, Number 1}
\ccoppy{\copyright ~2012 Aaron Duke}
\copnum{aaron.duke@uky.edu}
\ifapamodejou{ % JOU MODE
\note{Draft Date: \today{}} 
}{}
% \authornote{\hspace{-1.3em}
% \emph{Corresponding Author:}\\
% Aaron A. Duke \\
% Department of Psychology,
% University of Kentucky\\
% Lexington, KY 40506-0044, USA\\
% Email: \href{mailto:aaron.duke@uky.edu}{aaron.duke@uky.edu}
% }

% \title{Revisiting the Serotonin Deficiency Hypothesis\\ of Human Aggression: A Meta-analysis}
\title{Revisiting the Serotonin-Aggression Relation in Humans: A Meta-analysis}
% \title{The Decline Effect in Action: A Meta-analytic Review of Serotonin and Human Aggression}
\shorttitle{Serotonin and Human Aggression Revisited}


\abstract{%
The inverse relation between serotonin and human aggression is often portrayed as `reliable,' `strong,' and `well-established' despite decades of conflicting reports and widely recognized methodological limitations. In this systematic review and meta-analysis, we evaluate the evidence for and against the serotonin deficiency hypothesis of human aggression across four methods of assessing serotonin: cerebrospinal fluid levels of 5-hydroxyindoleacetic acid (CSF 5-HIAA), acute tryptophan depletion, pharmacological challenge, and endocrine challenge. Results across 175 independent samples and over 6,500 total participants were heterogeneous, but, in aggregate, revealed a small correlation between serotonin and aggression, anger, and hostility, $r$ = -.12. Potential methodological and demographic moderators largely failed to account for variability in outcomes. We discuss four possible explanations for the pattern of findings: 1) unreliable measures, 2) ambient correlational noise, 3) an unidentified higher-order interaction, and 4) a  selective serotonergic effect. Finally, we give four recommendations for advancing this important area of research: 1) acknowledge contradictory findings and avoid selective reporting practices,  2) focus on improving the reliability and validity of serotonin and aggression measures, 3) test for interactions involving personality and\slash or environmental moderators, and 4) revise the serotonin deficiency hypothesis to account for serotonin's functional complexity. 
} 
\keywords{serotonin, 5-HT, aggression, anger, hostility}

% Early evidence suggesting a possible link between serotonin and aggression 
% has sparked decades of exciting research into the neurochemical substrates of human aggression.
% With advances in measurement technology and a growing body of evidence, it has become clear 
% that previous notions of a simple deficiency in central serotonin levels acting as a catalyst 
% for aggressive and violent behavior are inadequate. 
% In spite of this, many studies continue to be designed around the several-decades-old theory 
% that a general deficiency in central serotonin levels leads people to be more aggressive. 
% The current meta-analysis reviews studies across four distinct methods of assessing central 
% serotoninergic functioning in humans (cerebrospinal fluid 5-HIAA, acute tryptophan depletion, 
% pharmacological challenge, and endocrine challenge) in order to empirically assess whether the 
% serotonin deficiency hypothesis of human aggression is valid. 
% Results across 175 independent samples and over 6,500 total participants were heterogeneous, 
% but, in aggregate, revealed a ``small'' correlation between serotonin and aggression, anger, and hostility,
% $r$ = -.12. 
% Numerous methodological and demographic variables largely failed to account for variability in outcomes. 
% We argue that researchers should move beyond the simple deficiency hypothesis in examining the effects of 
% serotonergic neurotransmission on aggression and that the continued prominence of this hypothesis may be partially 
% responsible for previous failures to clarify the serotonin--aggression link in humans. 
% Implications and possible explanations of some unexpected meta-analytic findings are discussed, 
% including the finding that other-report\slash observational measures of aggression 
% were \emph{positively} and significantly correlated to serotonergic functioning, $r$ = .17.