
% Our original motivation for reviewing the serotonin and aggression literature
% stemmed from an awareness of several studies in which serotonin was not 
% found to be positively correlated to human aggression
% \parencite[e.g.,][]{}. 
% We wanted to know how such studies fit into the greater literature -- 
% whether they are outliers or possibly suffer from methodological issues. 
% Further, we wanted to evaluate whether or not the serotonin deficiency hypothesis 
% is consistent with the majority of empirical findings in this area.
% In the course of preparing to conduct a meta-analysis on the serotonin--aggression 
% link in humans, we encountered what we believe to be three factors that 
% hinder our ability to evaluate the serotonin defienciy hypothesis of human aggression:
% 1) the complexity of serotonergic functioning,
% 2) problems in measuring aggression, and
% 3) problems measuring serotonin activity. 
 

% TELL THEM WHAT YOU'RE GOING TO TELL THEM.
The current investigation seeks to clarify the relation between serotonin and human aggression by
reviewing four of the primary methodologies used to assess
central serotonergic functioning in humans and meta-analytically synthesizing 
findings across these methodologies as they relate to 
aggression and other aggression-related constructs such as hostility and anger.
% We will begin by presenting our operating definition of aggression 
% and addressing some of the perennial difficulties in defining this construct.
% We will then review serotonin assessment methodologies and their limitations 
% along with some of the complexities in understanding the putative roles of the serotonin
% system in human behavior.
We believe that a rigorous and systematic approach to reviewing the evidence for
the serotonin deficiency hypothesis of human
aggression will provide much needed clarity and guidance for further refinement 
of serotonergic theories of aggression.
In so doing, we hope to provide the impetus to move this important area of research forward.







% In order to answer these questions, 
% The current investigation seeks to clarify the relation between serotonin and 
% human aggression by reviewing four of the primary methodologies used to assess
% central serotonergic functioning in humans and meta-analytically synthesizing 
% findings across these methodologies as they relate to 
% aggression and other aggression-related constructs such as hostility and anger.
% We chose to focus on four primary methodologies used to assess central 
% serotonergic functioning in humans. 
% 
% Prior to introducing our systematic meta-analysis of 
% four of the primary methodologies used to assess
% central serotonergic functioning in humans and meta-analytically synthesizing 
% findings across these methodologies as they relate to 
% aggression and other aggression-related constructs such as hostility and anger.
% 
% The fact is, the relationship between serotonergic functioning 
% and human aggression is equivocal -- not ``well established.'' 
% We present three arguments for why this is the case. 
% 1: Serotonergic functioning is complex and the predominate 
% hypothesis of the serotonin--aggression link in humans does not 
% take this complexity into account. 
% 2: Our ability to measure serotonergic functioning in humans is limited.
% 3: Our ability to measure human aggression is limited. 
% However, our 
% wht
% 
% 
% % noted that this v
% % in her widely read book, \emph{Biobehavioral Perspectives in Criminology},
% % called the inverse relationship between serotonin activity 
% % and aggression ``perhaps the most reliable finding in the history of psychiatry'' 
% % \parencite[15]{Fishbein2001}.
% % However, somehow, somewhere along the way of 
% % 
% % Somewhere, 
% 
% % with mixed results \parencite[e.g.,][]{Passamonti2012}. 
% % A growing number of contradictory findings and increasing awareness of methodological 
% % limitations cast greater and greater doubt as to whether the serotonin deficiency
% % hypothesis is a sufficient explanation for how serotonin influences aggressive behavior. 
% % However, the uncertainty surrounding the relation between serotonin and human aggression
% % is rarely acknowledged in the extant literature. 
% % A glance through ?? recent psychology textbooks revealed that ??
% % \parencite[][]{Bushman2010f, Anderson2003c}.
% % 
% % %
% % % 
% % % \parencite[e.g.,][]{Passamonti2012}, and yet
% % % we 
% % 
% % % and remains the most common hypothesis of serotonin's 
% % % role in pathological aggression \parencite[e.g.,][]{Coccaro2010b, Montoya2011, Passamonti2012, %
% % % Raine2008, Stadler2007, Steiger2004, Yanowitch2011}. 
% % However, it is not the case that the evidence supporting the serotonin deficiency hypothesis 
% % is uniformly supportive. 
% % %using the same paradigm as 
% % The 
% % 
% % 
% % 
% % with mixed results. 
% % % and yet the evidence for it remains confusing, disjointed, and incomplete. 
% % The truth is that we still do not know how serotonin influences aggression, 
% % and we do not even know if 
% % 
% % --
% % more fundamentally, we do not know if a serotonin deficiency reliably 
% % 
% % whether serotonergic deficiency is
% % 
% % 
% % 
% % However, the hypothesis 
% % is still under contentious debate. 
% % 
% % 
% % we still do not know if the hypothesis is valid. 
% % 
% % if the hypothesis is 
% % our current understanding of serotonin's relation to human aggression remains incomplete. 
% % This unc
% % If anything, we are \emph{less} certain that serotonin
% % is robustly related 
% % 
% % 
% % We have noticed a trend for researchers to selectively cite affirmatory studies
% % in this area without acknowledging the high degree of uncertainty that remains in our
% % understanding of serotonin's relation with human aggression. 
% % 
% % 
% % 
% % However, an alarmingly number of researchers 
% % % PEOPLE DON"T KNOW 
% % However, there is a general lack of awareness of the many 
% % % MISINFORMATION
% % % DISREPLICAITONS
% % % MEAUSREMENT
% % % COMPLEXITY
% % %
% % Disturbingly, 
% % % BLank and Blank calls this finding "one of the most robust" blah blah blah.
% % % Blah, blah, blah noted that .....
% % Claiming that the serotonin--aggression link in humans is %robust, beyond dispute, whatev...
% % requires a selective review of the literature. 
% % In fact, numerous studies have found nonexistent or 
% % contradictory findings \parencite[e.g.,][]{Tuinier96}. % Add more references here.
% 
% 
% The fact is, the relationship between serotonergic functioning 
% and human aggression is equivocal -- not ``well established.'' 
% We present three arguments for why this is the case. 
% 1: Serotonergic functioning is complex and the predominate 
% hypothesis of the serotonin--aggression link in humans does not 
% take this complexity into account. 
% 2: Our ability to measure serotonergic functioning in humans is limited.
% 3: Our ability to measure human aggression is limited. 
% 
% 
% The serotonin deficiency hypothesis of aggression, as limited as it is,
% may or may not be accurate.
% 
% We make the claim here that the serotonin--aggression link in 
% %
% However, scholarly works citing the support for a serotonin--aggression link in humans often
% fail to acknowledge the growing number of contradictory findings (e.g., ??, ??, ??). 
% Work in this domain has been critical in advancing our understanding of serotonergic functioning; 
% however, surprisingly, the central theory of a serotonin deficiency has not evolved along side 
% our understanding. 
% make the case for why the serotonin deficiency 
% In the present article, we argue against considering the serotonin--aggression link in humans
% robust. 
% make the case for why the serotonin deficiency 
% hypothesis is insufficient. 
% % NOT ROBUST, INSUFICIENT, UNTESTABLE
% In so doing, we hope to provide the impetus to move this important area of research forward.
% we make the case for why the current 
% %

% 
% % ARGUMENT: Basically, I'm saying that everyone says this thing is robust, and I'm like, I don't think it 
% % is and so why don't we test it. It turns out that its not that robust - plus it's a drastic oversimplification
% % of what is actually going on. 
% 
% 
% 
% But time has not treated the deficiency hypothesis well. 
% As the number of replications have mounted, more and more contradictory 
% findings challenge the sufficiency of the original hypothesis. 
% 
% 
% But, is the deficiency hypothesis supported by the evidence? 
% In spite of its popularity, there is 
% are some serious concerns about both the validity and 
% the utility of the serotonin deficiency hypothesis. 
% % UTILITY, VALIDITY, and CERTAINTY 
% % VALIDITY = contradictory findings
% % CERTAINTY = uncertainty in measurement
% % UTILITY = complexity of 5-HT 
% 
% 
% However, there is reason to question the utility of the serotonin deficiency hypothesis.
% An increased number of contradictory findings \parencite[e.g.,][]{Tuinier96} coupled with an
% increased awareness of methodological problems in measuring human serotonergic functioning and
% human aggression have left the validy of th e
% leaving the validity of the hypothesis in question and the precise nature of the 
% % serotonin--aggression link in humans unclear.
% 
% There have been an increasing number of contradictory findings \parencite[e.g.,][]{Tuinier96} 
% % However, there is reason to question whether or not the serotonin deficiency hypothesis 
% % is consistent with the majority of empirical findings relating to serotonin and human aggression.
% % A number of seemingly inconsistent and contradictory finding \parencite[e.g.,][]{Tuinier96} 
% % have yet to be resolved by proponents of the serotonin deficiency hypothesis, 
% % leaving the validity of the hypothesis in question and the precise nature of the 
% % serotonin--aggression link in humans unclear.
