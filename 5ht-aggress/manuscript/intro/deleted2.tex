
%%%%%%%
% ATD %
%%%%%%%
\subsubsection{Acute Tryptophan Depletion (ATD)}  
\begin{quotation}\noindent 
``\emph{Although ATD \ldots [is] important in the investigation of the monoamine systems, monoamine depletion does not directly decrease mood}'' \parencite{Ruhe2007}.\end{quotation}


% ATD MECHANISMS OF ACTION
\indent Acute tryptophan depletion represented the first broad attempt at experimentally manipulating central serotonin levels in humans \parencite{Young1985}. Acute tryptophan depletion relies on the assumption that modifying plasma levels of  \textsc{l}-tryptophan, the primary precursor of serotonin, will lead to corresponding  modifications in levels of brain serotonin \parencite{Hood05}. Because the blood brain barrier prevents peripheral serotonin from entering directly into the central nervous system, serotonin in the brain must be synthesized locally from its precursor tryptophan. Humans receive all of their tryptophan from dietary sources (Moore et al., \citeyear{Moore2000}) and therefore, dietary tryptophan intake can be experimentally manipulated to influence central serotonin availability. However, free plasma levels of tryptophan are relatively resilient to short periods of tryptophan-free dieting, with a typical reduction in only about 15--20\% of central serotonin availability \parencite{Delgado89}. Thus, while often used as an experimental control, tryptophan-free diets are less effective at reducing serotonin availability than acute tryptophan depletion. Acute tryptophan depletion relies on two additional physiological processes in order to reduce central serotonin availability \parencite[for an overview of the rationale and methodology of acute tryptophan depletion, see][]{Hood05}. The first is that tryptophan must compete with five other large neutral amino acids (LNAA) for active-transport across the blood brain barrier. Therefore, the ratio of free plasma tryptophan to other LNAAs determines the rate by which tryptophan can cross the blood brain barrier and be synthesized into serotonin. The tryptophan/LNAA ratio can be augmented by increasing available tryptophan relative to other LNAAs (e.g., through direct administration of tryptophan supplements), and can be attenuated (i.e., in acute tryptophan depletion) by increasing the availability of other LNAAs relative to tryptophan (e.g., through administration of non-tryptophan supplements including other LNAAs). The second process has to do with synthesis of proteins in the liver. When there is a dietary influx of amino acids, the liver will begin to synthesize various new proteins, many of which require tryptophan. If an amino-acid mixture (containing both LNAAs and non-LNAAs) is administered without any tryptophan included, the liver will use existing plasma tryptophan for protein-synthesis, thereby further reducing tryptophan's ability to be synthesized into serotonin.


% OPERATIONALIZATIONS AND MEASUREMENT
While a variety of different amino acid mixtures have been utilized in acute tryptophan depletion studies, they can be roughly divided into three categories: tryptophan-free (\texttt{T-}) amino acid mixtures, balanced (\texttt{B}) amino acid mixtures, and tryptophan-enhanced (\texttt{T+}) mixtures. Young, as the first to use acute tryptophan depletion in humans, standardized the 100 gram \texttt{T-} amino mixture with 15 amino acids in the proportion that are found in human breast milk minus the tryptophan \parencite{Young1985}. Young's balanced mixture consisted of adding 2.3 grams of tryptophan to the 100 gram mixture, while the \texttt{T+} mixture had 10.3 grams of tryptophan (i.e., a higher than normal ratio of tryptophan compared to the other amino acids). In these mixtures, the amino acids aspartic acid and glutamic acid are typically excluded over concerns about toxicity \parencite{Hood05}. Further, these amino acid mixtures are known for having a very ``unpalatable'' taste to them -- an issue that has been addressed differently by different researchers. Often something is added to the mixture to improve its taste (e.g., chocolate syrup). Also, researchers began administering amino acids containing sulfur separately from the rest of the mixture in the form of capsules, as these amino acids (methionine, cystine, and arginine) are considered the ``most unpalatable.''


% REVIEWS AND META-ANALYSES
Similar amino-acid depletion techniques have been adopted for other monoamines and a number of reviews have been conducted examining the effects of monoamine depletion on mood \parencite{Ruhe2007, Vanderdoes2001}, anxiety \parencite{Anderson1999}, psychiatric disorders (Bell, Hood, \& Nutt, \citeyear{Bell2005}; Booij, Van der Does, \& Riedel, \citeyear{Booij03}; Moore et al., \citeyear{Moore2002}; Reilly, McTavish, \& Young, \citeyear{Reilly1997}; Sobczak, Honig, Duinen, \& Riedel, \citeyear{Sobczak2002}), neural activation \parencite{Fusar2006}, memory \parencite{Sambeth2007}, executive functions \parencite{Mendelsohn2009}, and cognitive flexibility \parencite{Evers2007}. \citeauthor{Booij03} analyzed the literature on acute tryptophan depletion and  aggression and found several contradictory findings. These authors concluded that there was evidence for acute tryptophan depletion increasing aggression in individuals predisposed to act aggressively, but not in individuals with low trait levels of aggression. A recent systematic review of acute tryptophan depletion's effects on executive functioning concluded that the majority of findings indicate that acute tryptophan depletion does \emph{not} impair key executive functioning processes including response inhibition, decision making, planning, sustained attention, and set shifting \parencite{Mendelsohn2009}. It should be noted that impaired executive functioning has been argued to be important in predicting many forms of impulsive aggression \parencite[e.g.,][]{Hancock2010}. In the only relevant meta-analysis conducted on monoamine depletion studies, acute tryptophan depletion was found to have no effect on mood in healthy individuals, but to moderately decrease mood in individuals with a history of depression \parencite{Ruhe2007}. Finally, findings from a recent review of an alternate, but much less common approach to serotonin manipulation --- tryptophan augmentation --- have suggested similarly equivocal results with respect to mood and similarly negative results with respect to executive functioning \parencite{Silber2010}.
