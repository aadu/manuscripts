% \documentclass[man]{apa6}
% 
\usepackage[american]{babel}
\usepackage[backend=biber,style=apa,sortcites=true,sorting=nyt]{biblatex}
\DeclareLanguageMapping{american}{american-apa}
\usepackage{csquotes}
\addbibresource{/home/duke/Dropbox/rev5ht/paper/rev5ht.bib}
\usepackage{graphicx}
\usepackage{tikz}
\usepackage{amsmath}
\usepackage{amsfonts}
\usepackage{mathtools}
\usepackage{array}
\usepackage{mathptmx}
\usepackage{framed}
\usepackage{dcolumn}
\usepackage{multirow}
\newcolumntype{d}[1]{D{.}{.}{#1}}
\newcommand{\subscript}[1]{\ensuremath{_{\textrm{#1}}}}
\PassOptionsToPackage{hyphens}{url}\usepackage{hyperref} 
\usepackage{setspace}% http://ctan.org/pkg/setspace
\AtBeginEnvironment{tabular}{\singlespacing}% Single spacing in tabular environment
% http://tex.stackexchange.com/questions/44013/apa6-class-single-spaced-tables-for-manuscript
\usepackage{footmisc}
\renewcommand{\footnotelayout}{\doublespacing}
\definecolor{mygreen}{RGB}{0,87,13}
% \hypersetup{
% colorlinks=true,
% citecolor=mygreen,
% bookmarksopen=true
% }
\hypersetup{
colorlinks=true,
linkcolor=black,
citecolor=black,
filecolor=black,
urlcolor=black,
bookmarksopen=true
}


% ,
% pdfinfo={
%  Title={Bayesian Analysis of Parents' Drinking Motives and Their Children's Adjustment},
%  Author={Aaron A. Duke, M.S.}
% }, 
% pdfdisplaydoctitle=true
% }

% 1.   One minor suggestion to the authors might be “a collective within-lab factor analysis of constructs” prior to z transformation. However I do not believe this effort is necessary for the publication of this analysis.

\author{~} 
% \affiliation{\vspace{1mm}Department of Psychology\\University of Kentucky}
\affiliation{~}
\journal{Psychological Bulletin}
\volume{Volume 0, Number 1}
\ccoppy{\copyright ~2012 Aaron Duke}
\copnum{aaron.duke@uky.edu}
\ifapamodejou{ % JOU MODE
\note{Draft Date: \today{}} 
}{}
% \authornote{\hspace{-1.3em}
% \emph{Corresponding Author:}\\
% Aaron A. Duke \\
% Department of Psychology,
% University of Kentucky\\
% Lexington, KY 40506-0044, USA\\
% Email: \href{mailto:aaron.duke@uky.edu}{aaron.duke@uky.edu}
% }

% \title{Revisiting the Serotonin Deficiency Hypothesis\\ of Human Aggression: A Meta-analysis}
\title{Revisiting the Serotonin-Aggression Relation in Humans: A Meta-analysis}
% \title{The Decline Effect in Action: A Meta-analytic Review of Serotonin and Human Aggression}
\shorttitle{Serotonin and Human Aggression Revisited}


\abstract{%
The inverse relation between serotonin and human aggression is often portrayed as `reliable,' `strong,' and `well-established' despite decades of conflicting reports and widely recognized methodological limitations. In this systematic review and meta-analysis, we evaluate the evidence for and against the serotonin deficiency hypothesis of human aggression across four methods of assessing serotonin: cerebrospinal fluid levels of 5-hydroxyindoleacetic acid (CSF 5-HIAA), acute tryptophan depletion, pharmacological challenge, and endocrine challenge. Results across 175 independent samples and over 6,500 total participants were heterogeneous, but, in aggregate, revealed a small correlation between serotonin and aggression, anger, and hostility, $r$ = -.12. Potential methodological and demographic moderators largely failed to account for variability in outcomes. We discuss four possible explanations for the pattern of findings: 1) unreliable measures, 2) ambient correlational noise, 3) an unidentified higher-order interaction, and 4) a  selective serotonergic effect. Finally, we give four recommendations for advancing this important area of research: 1) acknowledge contradictory findings and avoid selective reporting practices,  2) focus on improving the reliability and validity of serotonin and aggression measures, 3) test for interactions involving personality and\slash or environmental moderators, and 4) revise the serotonin deficiency hypothesis to account for serotonin's functional complexity. 
} 
\keywords{serotonin, 5-HT, aggression, anger, hostility}

% Early evidence suggesting a possible link between serotonin and aggression 
% has sparked decades of exciting research into the neurochemical substrates of human aggression.
% With advances in measurement technology and a growing body of evidence, it has become clear 
% that previous notions of a simple deficiency in central serotonin levels acting as a catalyst 
% for aggressive and violent behavior are inadequate. 
% In spite of this, many studies continue to be designed around the several-decades-old theory 
% that a general deficiency in central serotonin levels leads people to be more aggressive. 
% The current meta-analysis reviews studies across four distinct methods of assessing central 
% serotoninergic functioning in humans (cerebrospinal fluid 5-HIAA, acute tryptophan depletion, 
% pharmacological challenge, and endocrine challenge) in order to empirically assess whether the 
% serotonin deficiency hypothesis of human aggression is valid. 
% Results across 175 independent samples and over 6,500 total participants were heterogeneous, 
% but, in aggregate, revealed a ``small'' correlation between serotonin and aggression, anger, and hostility,
% $r$ = -.12. 
% Numerous methodological and demographic variables largely failed to account for variability in outcomes. 
% We argue that researchers should move beyond the simple deficiency hypothesis in examining the effects of 
% serotonergic neurotransmission on aggression and that the continued prominence of this hypothesis may be partially 
% responsible for previous failures to clarify the serotonin--aggression link in humans. 
% Implications and possible explanations of some unexpected meta-analytic findings are discussed, 
% including the finding that other-report\slash observational measures of aggression 
% were \emph{positively} and significantly correlated to serotonergic functioning, $r$ = .17.
% \begin{document}
% \maketitle


\begin{quotation}\noindent
``\emph{The greatest enemy of knowledge is not ignorance, it is the illusion of knowledge.}''- Stephen Hawking
\end{quotation}


There has been considerable enthusiasm at pinpointing human aggression's neurochemical origins ever since Brown and colleagues' original finding that cerebrospinal fluid (CSF) levels of the principle serotonin metabolite, 5-hydroxyindoleacetic acid (5-HIAA), ``accounted for 80\% of the variance in aggression scores'' among a group of military men diagnosed with personality disorders  \parencite[133]{Brown79}. These results were interpreted to suggest that a deficiency in serotonin was largely responsible for these men's aggressive behavior. This \emph{serotonin deficiency hypothesis} of human aggression has been tested hundreds of times over the past several decades and remains the most common hypothesis of serotonin's role in pathological aggression \parencite[e.g.,][]{Montoya2011, Passamonti2012, Raine2008, Stadler2007, Steiger2004, Yanowitch2011}. One widely cited author called the inverse relationship between serotonin activity and aggression ``perhaps the most reliable finding in the history of psychiatry'' \parencite[15]{Fishbein2001}. Glancing through several recently published psychology and psychiatry textbooks reveals a concordant lack of controversy -- no mention of contradictory findings \parencite[e.g.,][]{Bushman2010f, Carlson2010, Higley2007, Hollander2008}.


One \emph{sine qua non} of science is a healthy skepticism, and with a little bit of effort one can find numerous examples of studies finding no relation, or a positive relation between serotonin and human aggression \parencite[e.g.,][]{Booij2010, Castellanos1994, Gardner1990, Germine1992, Handelsman1998, Koszycki1996, Lopez1996, Tuinier96, Wood2006}. Understanding of the relation between serotonin and human aggression will be advanced by revisiting this ``most reliable finding'' in a more inclusive and objective light. We begin this review by critically evaluting the evidence for the serotonin deficiency hypothesis across four methods of assessing serotonin: CSF 5-HIAA, acute tryptophan depletion,  pharmacological challenge, and endocrine challenge. We then introduce a comprehensive meta-analysis designed to test whether divergent findings in the literature can be explained by taking into account a variety of methodological and sample-level moderators. 


%%%%%%%%%%%%%%
% CSF 5-HIAA %
%%%%%%%%%%%%%%
\subsection{CSF 5-HIAA}
\begin{quotation}\noindent 
``\emph{There is too much overlap between the mean 5-HIAA levels of `normal' and `affected' populations for the term `low 5-HIAA levels' to have a scientifically legitimate meaning \ldots Thus, low 5-HIAA levels are inappropriate markers for increased `risk' for any specific psychiatric condition or behavior in the general population or in comparison among individuals.}'' 
\parencite[][29]{Balaban1996}.\end{quotation}


% REVIEWS and META-ANALYSES
Several reviews have been published examining the relation between 5-HIAA and aggressive, disruptive, or antisocial behaviors \parencite[e.g.,][]{Asberg94, Balaban1996, Moore2002, Tuinier96, Tuinier95}. The results from these reviews have been equivocal. The quotation at the beginning of this section was taken from an early meta-analysis \parencite{Balaban1996}, which found that 5-HIAA levels did not differ significantly between aggressive and unaggressive clinical groups and that both aggressive and unaggressive clinical groups had a small, but significant reduction in 5-HIAA levels when compared to nonclinical controls. Another early review \parencite{Tuinier96} concluded that only 8 out of 23 studies published at the time on 5-HIAA and aggression provided support for an inverse relation. Furthermore, these 8 studies were almost entirely restricted to personality-disordered young adult males such as those in Brown and colleague's original study \parencite{Brown79}. A 2002 meta-analysis of 16 studies looking at 5-HIAA and adult antisocial behavior (Moore et al., \citeyear{Moore2002}) revealed a  ``medium'' \parencite[][p. 157]{Cohen88} mean weighted effect size of $d =$ -0.45. Results revealed no significant effects of gender, alcoholism, suicidality, or whether violence was against persons or property. Age, however, was found to significantly moderate the relation between 5-HIAA and aggression, such that studies with young adult, antisocial individuals exhibited a much larger effect than studies with older antisocial individuals. There are a number of limitations in the two meta-analyses just mentioned that require further qualification. Both \citeauthor{Balaban1996} and Moore et al. (2002) failed to account for study quality, excluded studies with limited statistical information (e.g., studies reporting only that findings were ``non-significant''), and did not test, or attempt to account for, publication bias. Thus, there is reason to believe that these meta-analyses may have included a sample of studies biased towards finding statistically significant results.


% CRITICISMS
The failure to clarify the ambiguous relation between 5-HIAA and aggression may be in part due to difficulties in establishing reference levels of CSF 5-HIAA. CSF 5-HIAA has been shown to be moderated by age \parencite[e.g.,][]{Hedner1986, Seifert1980, Takeuchi2000}, gender \parencite[e.g.,][]{Blennow1993, Hagenfeldt1984}, height and weight \parencite[e.g.,][]{Blennow1993, Hartikainen1991, Stroembom1996}, physical activity \parencite{Eklundh2001, Nordin1996}, season \parencite[e.g.,][]{Brewerton1988, Hartikainen1991}, atmospheric pressure \parencite{Eklundh1994, Nordin1992}, and intra-spinal pressure \parencite{Eklundh2001a}, along with several psychiatric and neurological conditions \parencite[for a review of CSF 5-HIAA moderators see][]{Dhondt04}. One of the most serious confounding influences associated with measuring CSF 5-HIAA may be an inverse relation between CSF 5-HIAA levels and stress associated with the lumbar puncture procedure itself (Hill et al., \citeyear{Hill1999}). 
Despite CSF 5-HIAA's important historical role as the first biomarker used to support the serotonin deficiency hypothesis in humans, there remain misgivings about the reliability and validity of using CSF 5-HIAA as an index of central serotonergic functioning \parencite[e.g.,][]{Balaban1996, Hyland2008, Vegt2003}.
% Despite the popularity of using 5-HIAA as a measurement of central serotoninergic functioning and its important historical role as the first methodology used to support the serotonin deficiency hypothesis in humans, there remain misgivings about the reliability, validity, and utility of using CSF 5-HIAA as an index of central serotonergic functioning \parencite[e.g.,][]{Balaban1996, Hyland2008, Vegt2003}.



%%%%%%%
% ATD %
%%%%%%%
\subsection{Acute Tryptophan Depletion (ATD)}  
\begin{quotation}\noindent 
``\emph{Although ATD \ldots [is] important in the investigation of the monoamine systems, monoamine depletion does not directly decrease mood}'' \parencite{Ruhe2007}.\end{quotation}


% ATD MECHANISMS OF ACTION
\indent Acute tryptophan depletion represented the first broad attempt at experimentally manipulating central serotonin levels in humans \parencite{Young1985}. Acute tryptophan depletion relies on the assumption that modifying plasma levels of  \textsc{l}-tryptophan, the primary precursor of serotonin, will lead to corresponding  modifications in levels of brain serotonin \parencite{Hood05}. Because the blood brain barrier prevents peripheral serotonin from entering directly into the central nervous system, serotonin in the brain must be synthesized locally from its precursor tryptophan. Humans receive all of their tryptophan from dietary sources (Moore et al., \citeyear{Moore2000}) and therefore, dietary tryptophan intake can be experimentally manipulated to influence central serotonin availability. However, free plasma levels of tryptophan are relatively resilient to short periods of tryptophan-free dieting, with a typical reduction in only about 15--20\% of central serotonin availability \parencite{Delgado89}. Thus, while often used as an experimental control, tryptophan-free diets are less effective at reducing serotonin availability than acute tryptophan depletion. Acute tryptophan depletion relies on two additional physiological processes in order to reduce central serotonin availability \parencite[for an overview of the rationale and methodology of acute tryptophan depletion, see][]{Hood05}. The first is that tryptophan must compete with five other large neutral amino acids (LNAA) for active-transport across the blood brain barrier. Therefore, the ratio of free plasma tryptophan to other LNAAs determines the rate by which tryptophan can cross the blood brain barrier and be synthesized into serotonin. The tryptophan/LNAA ratio can be augmented by increasing available tryptophan relative to other LNAAs (e.g., through direct administration of tryptophan supplements), and can be attenuated (i.e., in acute tryptophan depletion) by increasing the availability of other LNAAs relative to tryptophan (e.g., through administration of non-tryptophan supplements including other LNAAs). The second process has to do with synthesis of proteins in the liver. When there is a dietary influx of amino acids, the liver will begin to synthesize various new proteins, many of which require tryptophan. If an amino-acid mixture (containing both LNAAs and non-LNAAs) is administered without any tryptophan included, the liver will use existing plasma tryptophan for protein-synthesis, thereby further reducing tryptophan's ability to be synthesized into serotonin.


% OPERATIONALIZATIONS AND MEASUREMENT
While a variety of different amino acid mixtures have been utilized in acute tryptophan depletion studies, they can be roughly divided into three categories: tryptophan-free (\texttt{T-}) amino acid mixtures, balanced (\texttt{B}) amino acid mixtures, and tryptophan-enhanced (\texttt{T+}) mixtures. Young, as the first to use acute tryptophan depletion in humans, standardized the 100 gram \texttt{T-} amino mixture with 15 amino acids in the proportion that are found in human breast milk minus the tryptophan \parencite{Young1985}. Young's balanced mixture consisted of adding 2.3 grams of tryptophan to the 100 gram mixture, while the \texttt{T+} mixture had 10.3 grams of tryptophan (i.e., a higher than normal ratio of tryptophan compared to the other amino acids). In these mixtures, the amino acids aspartic acid and glutamic acid are typically excluded over concerns about toxicity \parencite{Hood05}. Further, these amino acid mixtures are renowned for having a very ``unpalatable'' taste to them -- an issue that has been addressed differently by different researchers. Often something is added to the mixture to improve its taste (e.g., chocolate syrup). Also, researchers began administering amino acids containing sulfur separately from the rest of the mixture in the form of capsules, as these amino acids (methionine, cystine, and arginine) are considered the ``most unpalatable.''


% REVIEWS AND META-ANALYSES
Similar amino-acid depletion techniques have been adopted for other monoamines and a number of reviews have been conducted examining the effects of monoamine depletion on mood \parencite{Ruhe2007, Vanderdoes2001}, anxiety \parencite{Anderson1999}, psychiatric disorders (Bell, Hood, \& Nutt, \citeyear{Bell2005}; Booij, Van der Does, \& Riedel, \citeyear{Booij03}; Moore et al., \citeyear{Moore2002}; Reilly, McTavish, \& Young, \citeyear{Reilly1997}; Sobczak, Honig, Duinen, \& Riedel, \citeyear{Sobczak2002}), neural activation \parencite{Fusar2006}, memory \parencite{Sambeth2007}, executive functions \parencite{Mendelsohn2009}, and cognitive flexibility \parencite{Evers2007}. \citeauthor{Booij03} analyzed the literature on acute tryptophan depletion and  aggression and found several contradictory findings. These authors concluded that there was evidence for acute tryptophan depletion increasing aggression in individuals predisposed to act aggressively, but not in individuals with low trait levels of aggression. A recent systematic review of acute tryptophan depletion's effects on executive functioning concluded that the majority of findings indicate that acute tryptophan depletion does \emph{not} impair key executive functioning processes including response inhibition, decision making, planning, sustained attention, and set shifting \parencite{Mendelsohn2009}. It should be noted that impaired executive functioning has been argued to be important in predicting many forms of impulsive aggression \parencite[e.g.,][]{Hancock2010}. In the only relevant meta-analysis conducted on monoamine depletion studies, acute tryptophan depletion was found to have no effect on mood in healthy individuals, but to moderately decrease mood in individuals with a history of depression \parencite{Ruhe2007}. Finally, findings from a recent review of an alternate, but much less common approach to serotonin manipulation --- tryptophan augmentation --- have suggested similarly equivocal results with respect to mood and similarly negative results with respect to executive functioning \parencite{Silber2010}.



%%%%%%%%%%%%%%%%%%
% PHARM AND ENDO %
%%%%%%%%%%%%%%%%%%
\subsection{Pharmacological\slash Endocrine Challenge}
\begin{quotation}\noindent ``\emph{Caution seems to be indicated in drawing inferences about the functional status of central serotonergic neurotransmission from neuroendocrine responses elicited by challenge studies using serotonergic agents}'' \parencite[][213]{Murphy1996}.\end{quotation}


% INTRODUCE PHARMACOLOGICAL CHALLENGE
\indent Coinciding with the increase in availability of drugs that more selectively target various serotoninergic receptors, acute pharmacological challenge tests began to be applied to the study of human aggression in the late 1980s. In these studies, a serotonergic agent is administered and acute changes in aggression and related constructs are measured over the course of several hours. Similar to acute tryptophan depletion, pharmacological challenge studies allow for experimental manipulation of central serotonin levels and typically involve comparing mean responses across different drug dosage conditions. However, group level data derived from pharmacological challenge studies does not allow for measurement of individual differences in responsivity to manipulation of central serotonin levels.


% INTRODUCE ENDOCRINE CHALLENGE
The endocrine challenge paradigm, on the other hand, provides a putative index of individual responsivity to serotonergic agents. Evidence indicating increased hypothalamus-pituitary-adrenal activity following serotonergic pharmacological challenges led to the use of pituitary (e.g., prolactin, adrenocorticotropic hormone, growth hormone) and adrenal hormones (e.g., cortisol) as indices of individual serotoninergic functioning \parencite{Cowen1990}. Thus, endocrine challenge studies are pharmacological challenge studies in which hormonal endpoints are used as an index of central serotonergic functioning. Some concerns have been raised concerning the implications of differential hormonal responses following serotonergic challenges. For example, the quotation above was taken from \citeauthor{Murphy1996} in a review of ``neuroendocrine responses to serotonergic agonists as indices of the functional status of central serotonin neurotransmission in humans,'' in which a high level of discordance was found between different endocrine endpoints (e.g., prolactin and cortisol), temperature, and behavioral measures. Despite the popularity of endocrine challenge studies, information concerning their validity remains somewhat sparse.

 
% DRUGS
% D,L-FENFLURAMINE
A number of different pharmacological agents have been used in challenge studies, though the most popular choice has been d,l-fenflurmaine (d,l-FEN), the racemic mixture of two enantiomers, dextrofenfluramine (d-FEN) and levofenfluramine (l-FEN). The d-enantiomer (d-FEN) has also been used extensively in endocrine challenge studies. Some researchers \parencite[e.g.,][]{Coccaro2010} have noted that the l-isomer in d,l-FEN is problematic because it has been shown to have antagonistic effects on the dopaminergic system that can influence prolactin release \parencite[see][]{Ben-Jonathan1989, Coccaro1994, Crunelli1980}; however, the hormonal response to both d-FEN and d,l-FEN is thought to be ``very similar'' \parencite[436]{Coccaro2010}. Both d-FEN and d,l-FEN are thought to increase synaptic serotonin levels by increasing serotonin release from nerve terminals and by inhibiting its reuptake \parencite{Pine1997}.


%META-CHLOROPHENYLPIPERAZINE
\emph{Meta}-chlorophenylpiperazine ($m$CPP) was another early drug to receive widespread use in endocrine challenge tests. Similar to fenfluramine, $m$CPP has been shown both to increase serotonin terminal release as well as block its reuptake \parencite{Pettibone1984}. Drug discrimination studies have found that the stimulus effects of $m$CPP are primarily mediated by 5-HT$_\textrm{2C}$ receptors and to a lesser extent, 5-HT$_\textrm{1B}$ receptors \parencite{Callahan1994, Gatch2003, Gommans1998}. The discriminative effects of $m$CPP have also been shown to mimic those of d,l-FEN \parencite{Callahan1994}. Another phenylpiperazine, eltoprazine, which has been used in at least one pharmacological challenge study \parencite{Cherek1995}, acts as a 5-HT$_\textrm{1A}$ and 5-HT$_\textrm{1B}$ receptor agonist and a 5-HT$_\textrm{2C}$ receptor antagonist \parencite{Schipper1990}.


% IPSAPIRONE AND BUSPIRONE
Two azapirone class selective partial 5-HT$_\textrm{1A}$ receptor agonists, ipsapirone and buspirone, have been used by several serotonin challenge studies. Ipsapirone's affinity for the 5-HT$_\textrm{1A}$ receptor is thought to be 1000 fold greater than for the 5-HT$_\textrm{1D}$, 5-HT$_\textrm{2A}$, and 5-HT$_\textrm{2C}$ receptors \parencite{Hamon1988}, and is thought to act as a full agonist at somatodendritic 5-HT$_\textrm{1A}$ autoreceptors and a partial agonist at post-synaptic 5-HT$_\textrm{1A}$ receptors \parencite{Almeida2010}. Busprione has been used less often than ipsapirone in challenge studies probably because it has a less selective pharmacological profile \parencite[e.g.,][]{Moser1990}. 
%OTHER DRUGS
Other drugs that have been used include the racemic selective-serotonin reuptake inhibitor (SSRI) citalopram and its s-enantiomer escitalopram, the SSRIs paroxetine, fluvoxamine, fluoxetine, and zolmitriptan, as well as the amino acid tryptophan (see acute tryptophan depletion section above). 
 
 
While a diverse group of drugs have been used, the underlying theory behind pharmacological and endocrine challenge tests utilizing these drugs appears to be the same. These drugs are thought to lead to a relatively rapid increase in serotonin concentrations in the serotonergic synapses, which, in turn, is thought to activate the hypothalamus-pituitary-adrenal pathway \parencite{Berman2009, Kojima2003}. However, these drugs are clearly dissimilar along a number of important dimensions such as their specific affinities for different receptor types. This is problematic for a number of reasons, most notably being that the different serotonin receptors do not influence behavior in the same way \parencite{Quadros2009}.


%%%%%%%%%%%%%%%%%
% CURRENT STUDY %
%%%%%%%%%%%%%%%%%
\subsection{Current Study}
% SUMMARY OF INTRO - I.E., LIMITATIONS
The current meta-analytic investigation was designed to evaluate the evidence for the serotonin deficiency hypothesis of aggression in humans from the perspective of four major serotonin assessment methods. Reviewing the literature on these methods leads us to believe that there are limitations inherent in each method, particularly in CSF measurement of 5-HIAA, which may hinder the ability to draw consistent conclusions concerning the relation between serotonin and aggression. These methodological limitations are compounded by the complex, multi-faceted, and interactive nature of neural serotonergic systems \parencite[][]{Kerman2011, Murphy1996, Tuinier96} and the less-than-ideal reliability and validity of most aggression measures \parencite[see][]{Ferguson2012}. 
% Finally, high levels of measurement error coupled with a high rate of small-$n$ studies leaves the serotonin--aggression link particularly vulnerable to publication bias induced inflation \parencite[][]{Burdett2003, Levine2009, Sutton2000}.


% MAIN HYPOTHESIS
These concerns lead us to predict that, overall, we will find a limited (i.e., ``small'') relation between serotonin and aggression in spite of the large effect sizes found by early, small-$n$ studies \parencite[e.g.,][]{Brown79}. In order to make our primary hypothesis explicit, we define a ``small'' relation in the present context according to the conventions outlined by Cohen for correlation coefficient effect sizes (\citeyear{Cohen1992}).\footnote{Cohen identified $r = .1$ as a ``small'' effect, $r = .3$ as a ``medium'' effect, and $r = .5$ as a ``large'' effect (\citeyear{Cohen1992}, p. 157).} Liberally, we will interpret any statistically significant effect less than $r = .3$ as consistent with our primary hypothesis. Note that because correlation coefficients are signed and an \emph{inverse} correlation is expected, our actual hypothesis is that, overall, $r > -.3$ and $r < 0$. 


% SECONDARY HYPOTHESES
We do not expect findings on the relation between serotonin and aggression to be homogeneous either \emph{within} or \emph{between} methodologies used to assess serotonergic functioning. We hypothesize that inconsistent findings may be partially attributable to differences in study methodology and sample characteristics. Across methodologies, we expect the findings from CSF 5-HIAA studies to be the most inconsistent given the profound methodological concerns mentioned above; however, we do not have specific predictions on which methods will differ in which ways. Many researchers have argued that the serotonin deficiency hypothesis only holds for certain individuals, therefore, we propose to test the following additional hypotheses related to potential demographic moderators: 1) serotonin and aggression are inversely correlated in clinical samples more so than in healthy samples, 2) serotonin and aggression are inversely correlated primarily in individuals with histories of aggression \parencite[see][]{Booij03}, 3) serotonin and aggression are inversely correlated in males more so than in females \parencite[e.g.,][]{Tuinier96}, and 4) serotonin and aggression are inversely correlated in younger adults more so than in older adults (Moore et al., \citeyear{Moore2002}).


%#   #     #                                          
%#   ##   ## ###### ##### #    #  ####  #####   ####  
%#   # # # # #        #   #    # #    # #    # #      
%#   #  #  # #####    #   ###### #    # #    #  ####  
%#   #     # #        #   #    # #    # #    #      # 
%#   #     # #        #   #    # #    # #    # #    # 
%#   #     # ######   #   #    #  ####  #####   ####  
%#                                                    
\section{Methods}
\subsection{Inclusion and Exclusion Criteria}
To be included, studies needed to directly assess human aggression (or the related constructs of anger and hostility) along with serotonergic functioning using one of the four methodologies reviewed above (i.e., CSF 5-HIAA, acute tryptophan depletion, pharmacological challenge, or endocrine challenge). This review focused solely on other-directed aggression; therefore, studies that included only measures of self-directed harm were excluded.%
\footnote{For readers interested in serotonin's putative role in self-harm behaviors, including suicide, there are several excellent reviews and treatments on this topic \parencite[e.g.,][]{Lester1995, Mueller2004, Desmyter2011}.\\} 
To this point, we have not carefully distinguished between aggression, anger, and hostility. There are compelling conceptual reasons to treat anger, aggression, and hostility as a set of inter-related constructs respectively representing the \emph{affective}, \emph{behavioral}, and \emph{cognitive} aspects of the same underlying antagonism domain \parencite[see][]{Martin2000}. In practice, these constructs are hopelessly confounded in many aggression measures,%
\footnote{
The \emph{jingle fallacy} (applying the same label to different constructs; \nptextcite{Thordike1903}) and the \emph{jangle fallacy} (applying different labels to the same construct; \nptextcite{Kelley1927}) are rampant in aggression research. For example, the distinctions between subscales labeled ``exptrapunitive hostility,'' ``angry hostility,'' and ``anger expression'' are not discernable from their titles alone. One researcher's ``hostility'' is another's ``aggression'' or ``anger'' \parencite[e.g.,][]{Buchanan1999, Bushman2001p, Tremblay1991}. Manuscript titles such as ``Aggression Questionnaire hostility scale predicts anger in response to mistreatment'' can leave the uninitiated feeling slightly dizzy \parencite{Felsten1999}.} 
and for the sake of parsimony, we often use `aggression' to refer to aggression and the related constructs of hostility and anger.   


%%%%%%%%%%%%%%%%%%%
% SEARCH STRATEGY %
%%%%%%%%%%%%%%%%%%%
\subsection{Search Strategy}
% KEYWORD SEARCH
A comprehensive literature search was conducted for empirical studies of serotonin and aggression, anger, or hostility in humans across the following sources: PsycINFO, MEDLINE, CINAHL, System  for  Information  for  Grey Literature, Cochrane Central Register of Controlled Trials, and the Cochrane Database of Systematic Reviews. In each of these databases, study titles, abstracts, subjects, and keywords were searched from inception through December, 2011 using the following terms: \sc (aggress* or violen* or anger or angry or hostil*) and (((``5-hydroxy\-indole\-acetic acid'' or 5-hiaa or 5hiaa or ``mono\-amine metabo\-lite*'') and (``cerebro\-spinal fluid'' or csf)) or (``*tryptophan deplet*'' or atd or rtd or ``monoamine deplet*'' ) or (``serotonin reuptake inhibitor'' or ssri or ``endocrine challenge'' or ``pharmacological challenge'' or *chlorophenylpiperazine or mcpp or m-cpp or *paroxetine* or *fenfluramine* or *ipsapirone* or *citalopram* or *zolmitriptan* or *eltoprazine* or *fluvoxamine* or *escitalopram* or *fluoxetine*). \rm


% SYNTAX
Note that an asterisk here represents a wild card character utilized in many scholarly databases that will match with any number of letters, typically placed at the end of a word to allow for variations in word endings. For example, ``aggress*'' matches ``aggression,'' ``aggressive,'' ``aggressor,'' or any other word that began with the letters ``aggress.'' A number of limiters were used to narrow the search results from PsycINFO, MEDLINE, and CINAHL. Specifically, results from these sources were filtered to include only empirical studies with human participants. Citations and reference sections were reviewed in articles selected for inclusion. The \emph{curricula vit\ae} of authors with more than one included study were obtained  and reviewed for additional relevant studies when possible. Additionally, multiple requests for unpublished studies were made over relevant academic channels, including directly contacting numerous primary authors of serotonin and aggression studies.


%%%%%%%%%%%%%%%%
% EFFECT SIZES %
%%%%%%%%%%%%%%%%
\subsection{Effect Sizes}
% TRANSFORMATION
All effect sizes were converted to correlation coefficients \parencite{Borenstein09}. All $r$s were signed so that positive correlations indicate a positive relation between serotonergic functioning and aggression while negative correlations represents an inverse relation between serotonin and aggression. In order to account for dependencies introduced by studies with multiple-endpoints, $r$s were aggregated according to procedures outlined by \textcite{GleserOlkin2009}, which take into account the correlation between the multiple measures. When the correlation between measures was not available, a default correlation of .5 was used to aggregate within-study effects \parencite{Wampold1997}. Prior to the analyses, all $r$s underwent Fisher's variance stabilizing and normalizing transformation $r$-to-$z$ \parencite{FishersZ}.  Results were subsequently transformed back to $r$ prior to interpretation.


% VOTE COUNTING
Given a high prevalence of findings with insufficient data necessary to calculate an effect size, and the concerns of significant publication bias, effect size estimates for studies which reported only the direction and/or significance were calculated using maximum likelihood estimation (MLE) as outlined by \textcite{BushmanWang09}. This technique allows for an overall effect size estimate based upon observed significant differences ($\alpha = 0.05$) or observed differences where significance levels are not reported ($\alpha = 0.5$). In cases where detailed statistics were reported for statistically significant findings, but not for non-significant findings, the MLE-derived estimate was used when calculating aggregate study effects.


% \subsection{Coding}
% QUALITY 
Following the recommendation of \textcite{Orwin85}, confidence ratings were coded for study effect size statistics. Studies which reported raw data, bivariate correlation coefficients, means and standard deviations, mean standardized differences or other effect size statistics from which an effect size can be accurately derived were coded as involving ``minimal estimation.'' Complex, but complete statistics (e.g., partial correlation coefficients) or statistics where some uncertainty existed were coded as including ``moderate estimation.'' Finally, results derived from vote-counting procedures were coded as ``highly estimated.''


%%%%%%%%%%%%%%%%%
% DATA ANALYSES %
%%%%%%%%%%%%%%%%%
\subsection{Analyses}
Studies were weighted by the inverse of their variance \parencite{Shadish09}. Mean weighted effect sizes are presented for both fixed-effects and random-effects models with estimates of heterogeneity ($Q$ and $I^2$ statistics) derived from the fixed-effects model. Moderator analyses were conducted using a mixed-effects model when significant heterogeneity was evinced in the combined estimate \parencite[see][]{Lipsey01}. Random-effects and mixed-effects models were estimated using restricted maximum-likelihood estimation (REML) along with the \textcite{Knapp03} adjustment to account for uncertainty in the amount of residual heterogeneity.  All analyses were conducted in \texttt{R} using the metafor package \parencite{metafor}.  
% All analyses were conducted in \texttt{R} using the \texttt{metafor} package \parencite{metafor}.  


% MODERATOR ANALYSES
Analyses were conducted to assess whether potential methodological and\slash or sample-level variables explained significant variance in study outcomes. Methodological variables of interest included experimental design, sample size, quality of reported effect size, drug, dose, hormone, comparison group, control condition, and assessment instrument. Aggression measures were analyzed along four general dimensions: construct (anger, aggression, or hostility), source (self or other), modality (questionnaire, clinical interview, laboratory, criminal record, or observation), and temporal scope (state or trait). The following sample-level variables were also coded for moderator analyses: mean age, percent female, race/ethnicity, history of aggression, and presence of psychopathology. Methodological and demographic variables were not included in moderator analyses if reported in fewer than 50\% of studies. For the remaining potential moderator variables, multiple-imputation \parencite{Pigott09} was used to estimate missing values using the \texttt{mi} package \parencite{mi} for \texttt{R}. 


% PUBLICATION BIAS
The well-documented tendency for significant research findings to be published more often that non-significant findings has been highlighted as a problem in meta-analyses since its inception \parencite[see][]{Sutton09}. Fortunately, a variety of techniques have been developed to estimate the degree to which the ``file drawer problem'' \parencite[][638]{Rosenthal79} is distorting meta-analytic findings. One method relies on a visual inspection of the symmetry of a ``funnel plot'' where effect sizes are plotted against their variance or sample size. Because studies using larger sample sizes typically have less variance, their estimates of the population effect size should be more accurate than studies with smaller sample sizes or with larger variance in their findings.  Hence, in the measurement of a single population effect, one would expect multiple studies to form a ``funnel'' shape with less-accurate studies forming a broad, symmetrical distribution around the population effect and more-accurate studies forming a much tighter, symmetrical distribution around the population effect size. If this pattern is not evidenced, particularly in regard to studies with smaller samples or greater variance, there may be reason to believe that publication bias is significantly distorting one's findings. It is important to note that there are other possible explanations of small-study bias and therefore asymmetrical funnel plots should not be interpreted as providing conclusive evidence of publication bias. In order to assess for small-study bias in the present review, a series of funnel plots were created for each methodology used to assess 5-HT functioning. These plots were analyzed both visually and, when appropriate, statistically using Egger's regression test \parencite{Egger97}. Asymmetry tests are most appropriate when effect sizes are not significantly heterogeneous, there are at least 10 studies, the ratio of extreme variances is greater than 4, and at least some of the findings are statistically significant \parencite{Ioannidis2007}.


% \printbibliography
% \end{document}

