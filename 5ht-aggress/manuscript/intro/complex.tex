% \subsection{Serotonergic Complexities}
% \newcommand{\subscript}[1]{\ensuremath{_{\textrm{#1}}}}
% %INTRODUCE COMPLEXITY
% When the serotonin deficiency hypothesis of human aggression was originally conceived,
% our understanding of serotonergic neurotransmission was much simpler than it is today.
% Over the past several decades, 
% rapid advances in technology and copious amounts of research have led to a much clearer understanding of
% serotonin's role in the brain;
% however, with increased understanding has come an increased awareness of the 
% profound complexity of serotonin's role in regulating behavior.
% The implications of a putative deficit in 
% central serotonin functioning have also become less and less clear.
% We now know that ``serotonergic neurotransmission contributes to the regulation of 
% virtually every neurobiological function,
% including sensory, motor, cognitive, and higher order executive functions'' \parencite[35]{Kerman2011}. 
% We also know that ``the serotonergic system in the CNS has complex interactions with many,
% if not all other neurotransmitter systems in the brain'' \parencite[][p. 214]{Olivier2005}.
% Some researchers \parencite[see][]{Seo2008} have cited evidence linking elevated levels of dopamine 
% with aggression \parencite[e.g.,][]{Hadfield1983, Miczek1994, Netter1991, Tidey1996} and evidence that 
% serotonin modulates dopaminergic functioning \parencite[e.g.,][]{Millan1998}, to argue for the importance of
% \emph{interactions} between these neurotransmitter systems in predicting impulsive aggression.
% 
% 
% % 5-HT RECEPTOR DIVERSITY
% Not only does serotonin interact with other neurotransmitter systems, we also know that there are 
% ``several anatomically distinct [serotonergic]  pathways'' \parencite[212]{Murphy1996} and
% at least 7 families of serotonin receptors encompassing %at least 16 distinct types of
% well over a dozen different serotonin receptor types
% (5-HT\subscript{1A}, 5-HT\subscript{1B}, 5-HT\subscript{1D}, 5-HT\subscript{1E}, 
% 5-HT\subscript{1F}, 5-HT\subscript{2A}, 5-HT\subscript{2B}, 5-HT\subscript{2C}, 5-HT\subscript{3}, 
% 5-HT\subscript{4}, 5-HT\subscript{5A},
% 5-HT\subscript{6}, 5-HT\subscript{7A},
% 5-HT\subscript{7B}, 5-HT\subscript{7D}).
% These different receptors are often associated with fundamentally different biological functions \parencite{Kitson2007}. 
% The two receptor families most often linked to aggression are the 5-HT\subscript{1} and 5-HT\subscript{2} families;
% however, several preclinical studies have implicated 5-HT\subscript{3} 
% in the regulation of aggressive behavior as well
% \parencite[e.g.,][]{Carrillo2010, Cervantes2009, Cervantes2010, Ricci2004, Ricci2005}.
% Interpretation of a central serotonin deficit's impact on aggressive behavior is
% complicated by the fact that activation of the 5-HT\subscript{1A} and 5-HT\subscript{1B} 
% receptors is thought to \emph{attenuate} aggressive behaviors, while
% activation of the 5-HT\subscript{2A} and 5-HT\subscript{2C} receptors is thought to
% \emph{exacerbate} aggression \parencite{Quadros2009}.
% The serotonin deficiency hypothesis does not stipulate why a central deficit should affect 
% some serotonin receptors types more than others. 
% 
% 
% %POST-PRE SYNAPTIC DEBATE
% Further complexities are introduced when we have to 
% distinguish between \emph{post}synaptic activation of heteroreceptors 
% and \emph{pre}synaptic activation of autoreceptors \parencite{Olivier2005}.  
% \citeauthor{Boer2005} challenge the ``dogmatic view that 5-HT inhibits aggression'' with a 
% convincing demonstration using feral wild-type rats that the anti-aggressive effects of 
% 5-HT\subscript{1A} and 5-HT\subscript{1B} receptor agonists occur by \emph{inhibiting}
% serotonin release via their affinity with
% 5-HT\subscript{1A} and 5-HT\subscript{1B} somatodentritic and terminal autoreceptors.
% In other words, these authors suggest that aggression is associated with excess serotonergic activity, 
% which appears to be the opposite of what is implied by the serotonin deficiency hypothesis.
% %
% % FIXED
% %
% Of note also are the findings by some researchers 
% that human males with low or disrupted monoamine oxidase A (MAOA) expression 
% exhibit increased aggressive tendencies,
% particularly when raised in hostile environments 
% \parencite{Caspi2002, Foley2004, Frazzetto2007, Kim-Cohen2006, Manuck2000a, Weder2009, Widom2006}.
% Given MAOA's role in facilitating the metabolism of monoamines such as serotonin, impaired 
% MAOA expression indicates tonically \emph{increased} serotonin levels, which, according to 
% the deficiency hypothesis, should lead to \emph{reduced} aggression.  
% 
% 
% % TONIC vs. PHASIC SEROTONERGIC FUNCTIONING
% Finally, there remains considerable confusion between the effects of tonic verses phasic serotoninergic functioning. 
% \citeauthor{Boer2005}'s work suggests that normal offensive aggression is followed by brief spikes in
% serotonin levels \parencite[see also][]{Vegt2003},
% while an \emph{in vivo} study conducted by \citeauthor{Erp2000} showed decreases in prefrontal 
% serotonin release shortly after a fight,
% but not before or during it. 
% 
% 
% % SUMMARIZE 
% These complexities are not consistent with the presumed mechanistic
% simplicity of the serotonin deficiency hypothesis of aggression, and
% for this reason, many researchers have questioned its utility or 
% challenged its over-simplifications \parencite[e.g.,][]{Boer2005, Booij2010, Carrillo2009, Tuinier96}.
